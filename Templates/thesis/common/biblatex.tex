% ------------------------------------------------------------------------------
% Реализация библиографии пакетами biblatex и biblatex-gost с использованием движка biber
% ------------------------------------------------------------------------------

\usepackage{csquotes} % biblatex рекомендует его подключать. Пакет для оформления сложных блоков цитирования.

%%% Загрузка пакета с основными настройками %%%
\usepackage[
  backend=biber,% движок
  bibencoding=utf8,% кодировка bib файла
  sorting=none,% настройка сортировки списка литературы
  style=gost-numeric,% стиль цитирования и библиографии (по ГОСТ)
  language=autobib,% получение языка из babel/polyglossia, default: autobib % если ставить autocite или auto, то цитаты в тексте с указанием страницы, получат указание страницы на языке оригинала
  autolang=other,% многоязычная библиография
  clearlang=true,% внутренний сброс поля language, если он совпадает с языком из babel/polyglossia
  defernumbers=true,% нумерация проставляется после двух компиляций, зато позволяет выцеплять библиографию по ключевым словам и нумеровать не из большего списка
  sortcites=true,% сортировать номера затекстовых ссылок при цитировании (если в квадратных скобках несколько ссылок, то отображаться будут отсортированно, а не абы как)
  doi=false,% Показывать или нет ссылки на DOI
  isbn=false,% Показывать или нет ISBN, ISSN, ISRN
]{biblatex}

% ------------------------------------------------------------------------------
% Inclusion of bib-files
% ------------------------------------------------------------------------------

\addbibresource[label=main]{biblio/main.bib}

%http://tex.stackexchange.com/a/141831/79756
%There is a way to automatically map the language field to the langid field. The following lines in the preamble should be enough to do that.
%This command will copy the language field into the langid field and will then delete the contents of the language field. The language field will only be deleted if it was successfully copied into the langid field.
\DeclareSourcemap{ %модификация bib файла перед тем, как им займётся biblatex
    \maps{
        \map[overwrite]{ % добавляем ключевые слова, чтобы различать источники
            \perdatasource{biblio/main.bib}
            \step[fieldset=keywords, fieldvalue={bibliomain,bibliofull}]
        }    
        \map{% перекидываем значения полей language в поля langid, которыми пользуется biblatex
            \step[fieldsource=language, fieldset=langid, origfieldval, final]
            \step[fieldset=language, null]
        }
        \map[overwrite]{% перекидываем значения полей shortjournal, если они есть, в поля journal, которыми пользуется biblatex
            \step[fieldsource=shortjournal, final]
            \step[fieldset=journal, origfieldval]
        }
        \map[overwrite]{% перекидываем значения полей shortbooktitle, если они есть, в поля booktitle, которыми пользуется biblatex
            \step[fieldsource=shortbooktitle, final]
            \step[fieldset=booktitle, origfieldval]
        }
        \map[overwrite, refsection=0]{% стираем значения всех полей addendum
            \perdatasource{biblio/authorpapersVAK.bib}
            \perdatasource{biblio/authorpapersScopus.bib}
            \perdatasource{biblio/authorpapersWoS.bib}
            \perdatasource{biblio/authorpapers.bib}
            \perdatasource{biblio/authorconferences.bib}
            \step[fieldsource=addendum, final]
            \step[fieldset=addendum, null] %чтобы избавиться от информации об объёме авторских статей, в отличие от автореферата
        }
        \map[overwrite]{% перекидываем refbase в addendum, чтобы указать тип публикации (ВАК, Scopus, WoS) в конце ссылки
            \perdatasource{biblio/authorpapersVAK.bib}
            \perdatasource{biblio/authorpapersScopus.bib}
            \perdatasource{biblio/authorpapersWoS.bib}
            \step[fieldsource=refbase, final]
            \step[fieldset=addendum, origfieldval]
        }
        \map{% перекидываем значения полей numpages в поля pagetotal, которыми пользуется biblatex
            \step[fieldsource=numpages, fieldset=pagetotal, origfieldval, final]
            \step[fieldset=pagestotal, null]
        }
        \map{% если в поле medium написано "Электронный ресурс", то устанавливаем поле media, которым пользуется biblatex, в значение eresource.
            \step[fieldsource=medium,
            match=\regexp{Электронный\s+ресурс},
            final]
            \step[fieldset=media, fieldvalue=eresource]
        }
        \map[overwrite]{% стираем значения всех полей issn
            \step[fieldset=issn, null]
        }
        \map[overwrite]{% стираем значения всех полей abstract, поскольку ими не пользуемся, а там бывают "неприятные" латеху символы
            \step[fieldsource=abstract]
            \step[fieldset=abstract,null]
        }
        \map[overwrite]{ % переделка формата записи даты
            \step[fieldsource=urldate,
            match=\regexp{([0-9]{2})\.([0-9]{2})\.([0-9]{4})},
            replace={$3-$2-$1$4}, % $4 вставлен исключительно ради нормальной работы программ подсветки синтаксиса, которые некорректно обрабатывают $ в таких конструкциях
            final]
        }
        \map[overwrite]{% перекидываем значения полей howpublished в поля organization для типа online
            \step[typesource=online, typetarget=online, final]
            \step[fieldsource=howpublished, fieldset=organization, origfieldval]
            \step[fieldset=howpublished, null]
        }
        \map[overwrite]{% remove the 'urldate' field to avoid '(visited on...)'
            \step[fieldset=urldate,null]
        }        
    }
}
