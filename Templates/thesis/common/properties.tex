% Document properties

% ------------------------------------------------------------------------------
% Work name and its properties
% ------------------------------------------------------------------------------

\newcommand{\thesisTitle}              % Диссертация, название
{\todo{Thesis Title}}
\newcommand{\thesisSpecialtyNumber}    % Диссертация, специальность, номер
{\todo{01.04.02}}
\newcommand{\thesisSpecialtyTitle}     % 
{\todo{Applied Mathematics and Informatics}}
\newcommand{\thesisCity}               % City
{\todo{Moscow}}
\newcommand{\thesisYear}               % Year
{\todo{2019}}

% ------------------------------------------------------------------------------
% Author Name
% ------------------------------------------------------------------------------

\newcommand{\thesisAuthorLastName}{\todo{LastName}}
\newcommand{\thesisAuthorOtherNames}{\todo{FirstName}}
\newcommand{\thesisAuthorInitials}{\todo{F.\,L.}}

\newcommand{\thesisAuthor}             % Диссертация, ФИО автора
{%
    \texorpdfstring{% \texorpdfstring takes two arguments and uses the first 
    %for (La)TeX and the second for pdf
        \thesisAuthorLastName~\thesisAuthorOtherNames% так будет отображаться 
        %на титульном листе или в тексте, где будет использоваться переменная
    }{%
        \thesisAuthorLastName, \thesisAuthorOtherNames% эта запись для свойств 
        %pdf-файла. В таком виде, если pdf будет обработан программами для 
        %сбора 
        %библиографических сведений, будет правильно представлена фамилия.
    }
}

\newcommand{\thesisAuthorFL}             % First Name, Last Name
{%
    \texorpdfstring{% \texorpdfstring takes two arguments and uses the first 
    %for (La)TeX and the second for pdf
        \thesisAuthorOtherNames~\thesisAuthorLastName% так будет отображаться 
        %на титульном листе или в тексте, где будет использоваться переменная
    }{%
        \thesisAuthorOtherNames~\thesisAuthorLastName% эта запись для свойств 
        %pdf-файла. В таком виде, если pdf будет обработан программами для 
        %сбора 
        %библиографических сведений, будет правильно представлена фамилия.
    }
}

\newcommand{\thesisAuthorShort}        % Диссертация, ФИО автора инициалами
{\thesisAuthorInitials~\thesisAuthorLastName}

% ------------------------------------------------------------------------------
% Organization
% ------------------------------------------------------------------------------

\newcommand{\thesisOrganization}
{\todo{Federal State Autonomous Educational Institution of Higher 
Education\newline National Research University Higher School of Economics}}
\newcommand{\thesisOrganizationShort}{\todo{HSE}}

% ------------------------------------------------------------------------------
% Supervisors
% ------------------------------------------------------------------------------

%% \newcommand{\supervisorDead}{}           % Рисовать рамку вокруг фамилии
\newcommand{\supervisorFio}                 % Научный руководитель, ФИО
{\todo{Фамилия Имя Отчество}}
\newcommand{\supervisorRegalia}             % Научный руководитель, регалии
{\todo{уч. степень, уч. звание}} 
\newcommand{\supervisorFioShort}            % Научный руководитель, ФИО
{\todo{И.\,О.~Фамилия}}
\newcommand{\supervisorRegaliaShort}        % Научный руководитель, регалии
{\todo{уч.~ст.,~уч.~зв.}}

%% \newcommand{\supervisorTwoDead}{}        % Рисовать рамку вокруг фамилии
%% \newcommand{\supervisorTwoFio}           % Второй научный руководитель, ФИО
%% {\todo{Фамилия Имя Отчество}}
%% \newcommand{\supervisorTwoRegalia}       % Второй научный руководитель, 
%%%регалии
%% {\todo{уч. степень, уч. звание}}
%% \newcommand{\supervisorTwoFioShort}      % Второй научный руководитель, ФИО
%% {\todo{И.\,О.~Фамилия}}
%% \newcommand{\supervisorTwoRegaliaShort}  % Второй научный руководитель, 
%%%регалии
%% {\todo{уч.~ст.,~уч.~зв.}}

% To avoid conflict with beamer class use \providecommand
\providecommand{\keywords}%                 % keyword for PDF metadata
{}
