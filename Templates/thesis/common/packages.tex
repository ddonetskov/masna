\usepackage{etoolbox}       % for advanced tests
%\providebool{presentation}

% ------------------------------------------------------------------------------
% Geometry
% ------------------------------------------------------------------------------

\usepackage{geometry}           % for specifying the margins (later)

% ------------------------------------------------------------------------------
% Formatting
% ------------------------------------------------------------------------------

\ifnumequal{\number\value{draft}}{0}{  % if this is not the draft mode
%  \usepackage[final, shrink=45]{microtype}
}{}
%\usepackage{microtype}          % this is to get rid of 'Overfull \hbox...'

%\usepackage{indentfirst}        % Красная строка
\usepackage{soulutf8}           % support of underscore

% Page No above the page numbers in Contents
% http://tex.stackexchange.com/a/306950
\usepackage{afterpage}

%\usepackage{caption}
%\usepackage{subcaption}         % subcaptions in figures

% ------------------------------------------------------------------------------
% Math
% ------------------------------------------------------------------------------

\usepackage{amsthm,amsmath,amscd}       % AMS additions
\usepackage{amsfonts,amssymb}           % AMS additions
\usepackage{mathtools}          % adds multlined

\usepackage{cancel}             % four different modes of striking through
\usepackage{dsfont}
\usepackage{icomma}             % Smart comma: $0,2$ --- число, $0, 2$ --- 

\usepackage{physics}            % implementation of \abs and \norm
\usepackage{xfrac}              % a better maintained replacement for nicefrac

\DeclareMathOperator*{\D}{\mathbb{D}}   % the dispersion symbol
\DeclareMathOperator*{\E}{\mathbb{E}}   % the expectation symbol

\DeclareMathOperator*{\N}{\mathbb{N}}   % the set of natural numbers
\DeclareMathOperator*{\R}{\mathbb{R}}   % the set of real numbers
\DeclareMathOperator*{\Z}{\mathbb{Z}}   % the set of integers

% e = 2.71...
\newcommand{\e}{\mathrm{e}}

% sign of independence
% \vDash can also be used instead of \models
% \raisebox{}{} is required for vertical alignment
\newcommand{\independent}{
  \raisebox{0.05em}{\rotatebox[origin=c]{90}{$\models$}}
}

% ------------------------------------------------------------------------------
% Fonts settings
% ------------------------------------------------------------------------------

% settings for the font size 14 pt
% should be called before fontspec or polyglossia
%\newlength{\curtextsize}
%\newlength{\bigtextsize}
%\setlength{\bigtextsize}{13.9pt}

%\makeatletter
%\show\f@size                                       % неплохо для отслеживания, 
%%но вызывает стопорение процесса, если документ компилируется без команды  
%%-interaction=nonstopmode
%\setlength{\curtextsize}{\f@size pt}
%\makeatother

%\usepackage{polyglossia}        % multilanguage support, it loads fontspec also

% ------------------------------------------------------------------------------
% Lists
% ------------------------------------------------------------------------------

\usepackage{enumitem}           % to regulate space between items

% taken from https://tex.stackexchange.com/questions/300340/topsep-itemsep-partopsep-and-parsep-what-does-each-of-them-mean-and-wha
% \topsep    = vertical space added above and below the list.
% \partopsep = vertical space added above and below the list, but only if the 
% list starts a new paragraph.
% \itemsep   = vertical space added after each item in the list.
% \parsep    = vertical space added after each paragraph in the list.
\setlist[itemize]{topsep=2pt,partopsep=2pt,itemsep=2pt,parsep=0pt}

% ------------------------------------------------------------------------------
% Tables
% References
% - useful list of various environments: 
%https://tex.stackexchange.com/questions/214840/array-table-tabular-tabularx-longtable-supertabular-longtabu
% 
% ------------------------------------------------------------------------------

\usepackage{array}              % this is to automatically break longer lines 
%of text within cells,
% define fixed-width columns
\usepackage{booktabs}           % provides commands to produce heavier lines as 
%table frame
% (\toprule, \bottomrule) and lighter lines within a table
% (\midrule).
\usepackage{multirow}           % spanning columns across multiple rows
\usepackage{makecell}           % allows different formats inside cells

\renewcommand\arraystretch{1.5} % “stretches” the table vertically, increases 
%space between rows 
% \renewcommand{\tabcolsep}{3pt}  % adjust the space between columns

\renewcommand\theadalign{bc}
\renewcommand\theadfont{\bfseries}
\renewcommand\theadgape{\Gape[4pt]}
\renewcommand\cellgape{\Gape[4pt]}

\usepackage{longtable,ltcaption} % длинные таблицы
\usepackage{tabu, tabulary}      % таблицы с автоматически подбирающейся шириной столбцов

% ------------------------------------------------------------------------------
% Figures, graphics
% ------------------------------------------------------------------------------

% btw, graphicx is automatically loaded by memoir
\usepackage{graphicx}       

% ------------------------------------------------------------------------------
% Links
% ------------------------------------------------------------------------------

\usepackage{bookmark}

% hyperref usually needs to be loaded last
\usepackage[dvipsnames, table, hyperref]{xcolor}    % Совместимо с tikz
% btw, url is automatically loaded by memoir
\usepackage{url}

% ------------------------------------------------------------------------------
% Counters
% ------------------------------------------------------------------------------

\usepackage[figure,table]{totalcount}               % Счётчик рисунков и таблиц
\usepackage{totcount}                               % Пакет создания счётчиков на основе последнего номера подсчитываемого элемента (может требовать дважды компилировать документ)
\usepackage{totpages}                               % Счётчик страниц, совместимый с hyperref (ссылается на номер последней страницы). Желательно ставить последним пакетом в преамбуле
